\documentclass[12pt,letterpaper]{JHEP3}
\pdfoutput=1
\usepackage{epsfig}
\usepackage{showkeys}
\usepackage{subcaption}
\usepackage{graphicx}
\usepackage[english]{babel}
%\usepackage[utf8]{inputenc}
%\usepackage{mathtools}
%\usepackage{amsmath}
%\usepackage{tikz}
%\usepackage{subfigure}
%\usepackage{color}
%\input{rgb}
\usepackage{slashed}
\usepackage{float}
\usepackage{fancyhdr}
\usepackage{color}
\usepackage{amsfonts}%
\usepackage{amssymb}%
\usepackage{listings}
\usepackage{color}
\usepackage{feynmp-auto}
\DeclareGraphicsRule{*}{mps}{*}{} % for being able to read the produced file
\definecolor{mygreen}{rgb}{0,0.6,0}
\definecolor{mygray}{rgb}{0.5,0.5,0.5}
\definecolor{mymauve}{rgb}{0.58,0,0.82}
\lstset{ %
  backgroundcolor=\color{white},   % choose the background color; you must add \usepackage{color} or \usepackage{xcolor}
  basicstyle=\footnotesize,        % the size of the fonts that are used for the code
  breakatwhitespace=false,         % sets if automatic breaks should only happen at whitespace
  breaklines=true,                 % sets automatic line breaking
  captionpos=b,                    % sets the caption-position to bottom
  commentstyle=\color{mygreen},    % comment style
  deletekeywords={...},            % if you want to delete keywords from the given language
  escapeinside={\%*}{*)},          % if you want to add LaTeX within your code
  extendedchars=true,              % lets you use non-ASCII characters; for 8-bits encodings only, does not work with UTF-8
  frame=single,                    % adds a frame around the code
  keepspaces=true,                 % keeps spaces in text, useful for keeping indentation of code (possibly needs columns=flexible)
  keywordstyle=\color{blue},       % keyword style
  language=Octave,                 % the language of the code
  morekeywords={*,...},            % if you want to add more keywords to the set
  numbers=left,                    % where to put the line-numbers; possible values are (none, left, right)
  numbersep=5pt,                   % how far the line-numbers are from the code
  numberstyle=\tiny\color{mygray}, % the style that is used for the line-numbers
  rulecolor=\color{black},         % if not set, the frame-color may be changed on line-breaks within not-black text (e.g. comments (green here))
  showspaces=false,                % show spaces everywhere adding particular underscores; it overrides 'showstringspaces'
  showstringspaces=false,          % underline spaces within strings only
  showtabs=false,                  % show tabs within strings adding particular underscores
  stepnumber=2,                    % the step between two line-numbers. If it's 1, each line will be numbered
  stringstyle=\color{mymauve},     % string literal style
  tabsize=2,                       % sets default tabsize to 2 spaces
  title=\lstname                   % show the filename of files included with \lstinputlisting; also try caption instead of title
}
\usepackage{tcolorbox}
\usepackage[framemethod=TikZ]{mdframed}
\graphicspath{{figures/}} % Location of the graphics files
\usepackage{booktabs} % Top and bottom rules for table
\usepackage[font=small,labelfont=bf]{caption} % Required for specifying captions to tables and figures
\usepackage{amsfonts, amsmath, amsthm, amssymb} % For math fonts, symbols and environments
\usepackage{wrapfig} % Allows wrapping text around tables and figures

\usepackage{cite}


\graphicspath{{Figures/}}

\title{Interferometer Shifts in Supernova Explosions: Note}

\author{
}

\abstract{
}

\begin{document}

\section{Calculation}

\subsection{Before Shell Crossing}

Metric:
\begin{equation}
ds^2 = -\left(1-\frac{2M_{0}}{r}\right)dt^2 + \frac{dr^2}{1-\frac{2M_{0}}{r}}+r^2d\Omega_2^2~.
\end{equation}
Two end-points of an interferometer arm (ignore acceleration):
\begin{eqnarray}
r_1(t) &=& r_{E}~, \ \ \ r_2(t) = r_{E}+d~.
\end{eqnarray}
$(M_{0}/r_E)$ and $(d/r_E)$ are small numbers. An outgoing null-ray:
\begin{eqnarray}
t-t_0 &=& r-r_0 + 2M_{0}\ln\frac{r-2M_{0}}{r_0-2M_{0}} \\ \nonumber
&=& d + 2M_{0}\frac{d}{r_E} -M_{0}\frac{d}{r_E}
\left( \frac{d}{r_E}-\frac{4M_{0}}{r_E} \right)~.
\label{eq-dt1}
\end{eqnarray}
In the last step we plug in $r_0=r_E$, $r = r_E+d$, and expanded by large $r_E$ to the second order. We will see that this is at least necessary later. Likewise, an incoming null-ray:
\begin{eqnarray}
t-t_0 &=& r_0-r + 2M_{0}\ln\frac{r_0-2M_{0}}{r-2M_{0}} \\ \nonumber
&=& d + 2M_{0}\frac{d}{r_E} - M_{0}\frac{d}{r_E}
\left( \frac{d}{r_E}-\frac{4M_{0}}{r_E} \right) ~.
\end{eqnarray}
Here we used $r_0 = r_E+d$, $r = r_E$, and again expanded to the first order.

Thus the total coordinate time it takes for a light ray to come back at $r_1$ is
\begin{equation}
\Delta t = 2d + 4M_{0}\frac{d}{r_E} - 2M_{0}\frac{d}{r_E}
\left( \frac{d}{r_E}-\frac{4M_{0}}{r_E} \right)~.
\end{equation}
The total proper time is
\begin{eqnarray}
\Delta \tau &=& \sqrt{1-\frac{2M_{0}}{r_E}}\Delta t  \\ \nonumber
&=& 2d \left(1-\frac{M_{0}}{r_E} - \frac{M_{0}^2}{2r_E^2}\right) 
\left[ 1+\frac{2M_{0}}{r_E} - \frac{M_{0}}{r_E}\left( \frac{d}{r_E} - \frac{4M_{0}}{r_E} \right) \right] \\ \nonumber
&=& 2d
\left(1 + \frac{M_{0}}{r_E} + \frac{3}{2}\frac{M_{0}^2}{r_E^2} - \frac{M_{0}d}{r_E^2}\right)~.
\end{eqnarray}

\subsection{After Shell Crossing}

Metric:
\begin{equation}
ds^2 = -\left(1-\frac{2M_{1}}{r}\right)dt^2 + \frac{dr^2}{1-\frac{2M_{1}}{r}}+r^2d\Omega_2^2~,
\end{equation}
with $(M_{0}-M_{1})=\delta M$ the neutrino shell mass. Assuming that $r_1$ picks crosses the shell at $t=0$, we have
\begin{eqnarray}
r_1(t) &=& r_E - \frac{\delta M}{r_E}t~.
\end{eqnarray}

On the other hand, $r_2$ crosses the shell later. We can only derive that by first knowing the outgoing null-ray:
\begin{eqnarray}
t-t_0 = r-r_0 + 2M_{1}\ln\frac{r-2M_{1}}{r_0-2M_{1}}~.
\label{eq-null1out}
\end{eqnarray}
Plugging $t_0=0$, $r_0 = r_1(0) = r_E$, $r = r_2 = r_E+d$, we get the crossing time:
\begin{eqnarray}
t_2 &=& d + 2M_1\ln \frac{r_E+d-2M_1}{r_E-2M_1} \\ \nonumber
&=& d + 2M_1\frac{d}{r_E} - M_1\frac{d}{r_E}\left( \frac{d}{r_E} - \frac{4M_1}{r_E} \right)~.
\end{eqnarray}
Then we have
\begin{eqnarray}
r_2(t) &=& r_E + d - \frac{\delta M}{r_E+d}(t-t_2)
\\ \nonumber
&=& r_E + d - \frac{\delta M}{r_E}(t-d) + \frac{\delta M}{r_E}\frac{d}{r_E}(t-d)
+\frac{2\delta M M_1}{r_E^2}d~.
\end{eqnarray}

Now, for a light ray that starts from $r_1$ at a later time $t_0$, we solve where $r_2$ crosses the null ray, Eq.~(\ref{eq-null1out}).
\begin{eqnarray}
r_0 &=& r_1(t_0) = r_E - \frac{\delta M}{r_E}t_0~, \\
r &=& r_2(t) = r_E + d - \frac{\delta M}{r_E}(t-d) + \frac{\delta M}{r_E}\frac{d}{r_E}(t-d+2M_1)~.
\end{eqnarray}
A further approximation we may use is $t_0\gg d\approx(t-t_0)$. Thus when a term is already second order, we ignore the difference between $t$ and $t_0$. We can then start to solve the crossing time at $r_2$, while carefully keeping all the expansions up to the second order.
\begin{eqnarray}
\Delta t_{out}\equiv t - t_0 &=& d - \frac{\delta M}{r_E}(t - t_0 - d)
+ \frac{\delta M}{r_E}\frac{d}{r_E}(t-d+2M_1) \label{eq-SolveCross} \\ \nonumber
& & + 2M_{1}\ln\left(1+\frac{d}{r_E}-\frac{2M_{1}}{r_E} -\frac{\delta M(t-d)}{r_E^2} \right) 
\\ \nonumber
& & - 2M_{1}\ln\left(1-\frac{2M_{1}}{r_E} -\frac{\delta Mt_0}{r_E^2} \right)
\\ \nonumber
&=& d - \frac{\delta M}{r_E}(t - t_0 - d)
+ \frac{\delta M}{r_E}\frac{d}{r_E}(t_0+2M_1) \\ \nonumber
& & + 2M_{1} \left[ \frac{d}{r_E}-\frac{2M_{1}}{r_E} -\frac{\delta M(t-d)}{r_E^2}  - \frac{1}{2} \left( \frac{d}{r_E}-\frac{2M_{1}}{r_E} \right)^2 \right] \\ \nonumber
& & - 2M_{1} \left( -\frac{2M_{1}}{r_E} -\frac{\delta Mt_0}{r_E^2} - \frac{2M_{1}^2}{r_E^2} \right)~.
\end{eqnarray}

This leads to
\begin{eqnarray}
\Delta t_{out} \left(1 + \frac{\delta M}{r_E}\right) &=& 
d + 2M_{1}\frac{d}{r_E} - M_{1}\frac{d}{r_E}
\left( \frac{d}{r_E}-\frac{4M_{1}}{r_E} \right) \\ \nonumber
& & + \frac{\delta M}{r_E}d + \frac{\delta Md}{r_E^2} (t_0+2M_1)~.
\end{eqnarray}
Compare this to Eq.~(\ref{eq-dt1}), there are only three differences, and two of them have clear physical meanings. The last term on the right-hand-side is the total extra distance between the two end-points, since their velocities are slightly different. The extra factor on the left-hand-side comes from the fact that both end-points are falling toward the center with roughly identical velocity, so it takes a light ray less time to go from inside to outside. On the other hand, the outside endpoint did not pick up this velocity at $t=0$, but at $t\sim d$. That leads to the second last term which will cancel the previous factor at the leading order. The last two things will of course reversed in the reflection, while the distance increase stays the same. 

An incoming null-ray:
\begin{eqnarray}
t_0 - t = r-r_0 + 2M_{1}\ln\frac{r-2M_{1}}{r_0-2M_{1}}~.
\end{eqnarray}
The end points.
\begin{eqnarray}
r_0 &=& r_2(t_0) = r_E +d - \frac{\delta M}{r_E}(t_0-d) + \frac{\delta M}{r_E}\frac{d}{r_E}(t_0-d+2M_1) ~, \\
r &=& r_1(t) = r_E - \frac{\delta M}{r_E}t~.
\end{eqnarray}

\begin{eqnarray}
\Delta t_{in} \equiv 
t-t_0 &=& d + \frac{\delta M}{r_E}(t-t_0+d) + \frac{\delta M}{r_E}\frac{d}{r_E}(t_0-d+2M_1)
\\ \nonumber
& & +2M_1\ln\left( 1+\frac{d}{r_E} - \frac{2M_1}{r_E} - \frac{\delta M}{r_E^2}(t_0-d) \right)
\\ \nonumber
& & -2M_1\ln\left( 1 - \frac{2M_1}{r_E} - \frac{\delta M}{r_E^2}t \right)
\end{eqnarray}
Compare this with Eq.~(\ref{eq-SolveCross}), the log terms are identical up to the higher order difference in $t$ and $t_0$ which we can ignore.
\begin{eqnarray}
\Delta t_{in}\left(1-\frac{\delta M}{r_E}\right) &=& 
d + 2M_{1}\frac{d}{r_E} - M_{1}\frac{d}{r_E}
\left( \frac{d}{r_E}-\frac{4M_{1}}{r_E} \right) \\ \nonumber
& & + \frac{\delta M}{r_E}d + \frac{\delta Md}{r_E^2} (t_0-d+2M_1)~.
\end{eqnarray}

Combining them, the total duration in the coordinate time is
\begin{eqnarray}
\Delta t &=& \Delta t_{out} + \Delta t_{in} \\ \nonumber
&=& d \left(1 - \frac{\delta M}{r_E} + \frac{\delta M^2}{r_E^2} \right) 
\left[ 1 + \frac{2M_{1}}{r_E} + \frac{\delta M}{r_E} -\frac{M_{1}}{r_E}\left( \frac{d}{r_E}-\frac{4M_{1}}{r_E} \right) + \frac{\delta M (t_0+2M_1)}{r_E^2} \right] \\ \nonumber 
&+& d \left(1 + \frac{\delta M}{r_E} + \frac{\delta M^2}{r_E^2} \right) 
\left[ 1 + \frac{2M_{1}}{r_E} + \frac{\delta M}{r_E} -\frac{M_{1}}{r_E}\left( \frac{d}{r_E}-\frac{4M_{1}}{r_E} \right)  + \frac{\delta M (t_0-d+2M_1)}{r_E^2} \right]  \\ \nonumber
&=& 2d \left[ 1 + \frac{2M_{1}}{r_E} + \frac{\delta M}{r_E} -\frac{M_{1}}{r_E}\left( \frac{d}{r_E}-\frac{4M_{1}}{r_E} \right) + \frac{\delta M (t_0+2M_1-d/2)}{r_E^2} +\frac{\delta M^2}{r_E^2} \right]~.
\end{eqnarray}



We also need to pay more attention while converting this to proper time.
\begin{eqnarray}
\Delta \tau &=& \int_{t_0}^{t_0+\Delta t}
 \sqrt{\left(1-\frac{2M_{1}}{r_1}\right)dt^2 - \left(1-\frac{2M_{1}}{r_1}\right)^{-1}dr^2} \nonumber	
 \\ 
&=& \int_{t_0}^{t_0+\Delta t} 
\sqrt{\left(1 - \frac{2M_{1}}{r_E - \frac{\delta M}{r_E}t}\right) - \frac{\delta M^2}{r_E^2} }
~dt \\ \nonumber
&=& \left(1 - \frac{M_{1}}{r_E} - \frac{\delta M^2}{2r_E^2} - \frac{M_{1}^2}{2 r_E^2} \right)
\Delta t \\
&=& 2d \left( 1 + \frac{M_1}{r_E} + \frac{\delta M}{r_E} -\frac{M_1d}{r_E^2}+ \frac{3}{2} \frac{M_1^2}{r_E^2}+ \frac{\delta M (t_0+2M_1-d/2)}{r_E^2} + \frac{\delta M^2}{2r_E^2} \right). \label{123}
\end{eqnarray}
Compare with before shell-crossing:
\begin{eqnarray}
\Delta \tau &=& 2d
\left(1 + \frac{M_{0}}{r_E} + \frac{3}{2}\frac{M_{0}^2}{r_E^2} - \frac{M_{0}d}{r_E^2}\right)~.
\end{eqnarray}
We see that $0$th order, $1=1$, check. $1$st order, $M_0 = M_1+\delta M$, check. $2$nd order there are many terms. First we check the terms independent of $d$ and $t_0$.
\begin{eqnarray}
\frac{3}{2} M_1^2+2\delta M M_1 + \frac{\delta M^2}{2} = \frac{3}{2} M_0^2	\label{125}
\end{eqnarray}

The remaining terms are

\begin{equation}
	-\frac{M_0 d}{r_E^2} = - \frac{M_1 d}{r_E^2} + \frac{\delta M t_0}{r_E^2} - \frac{d \delta M}{2r_E^2}
\end{equation}

\begin{tcolorbox}
If I solve this for $t_0$ it just gives $t_0 = - \frac{d}{2}$. Still not entirely clear to me what this means:p
\end{tcolorbox}

\underline{NOTE}: if we now take $\frac{\Delta M}{r_e} \rightarrow \frac{\Delta M}{r_e} \left( 1 - \frac{2M_1}{r_e} \right)^\frac{1}{2}$, the only change is in the $\frac{2 \Delta M M_1}{r^2_E}$ term in Eq (\ref{123}) , it just becomes  $\frac{3 \Delta M M_1}{r^2_E}$ and similarly $2 \delta M M_1$ becomes $3 \delta M M_1$ in Eq (\ref{125}) Using this information we can now find what the difference between proper times is. Already we have checked that there is no change upto first order in $\frac{1}{r_E}$, the change in second order is

\begin{tcolorbox}
\begin{equation}
	\Delta \tau_{final} - \Delta \tau_{initial} = 2d \left( \frac{1}{2} \frac{d \delta M}{r_E^2} + \frac{3}{2r_E^2} (M_1^2 - M_0^2) + \frac{3M_1 \delta M}{r_E^2} + \frac{\delta M^2}{2r_E^2} + \frac{\delta M t_0}{r^2_E} \right)
\end{equation}
\end{tcolorbox}

so there are a bunch of constant terms and a term that grows linearly with $t_0$ the time since shell crossing. As expected this goes to zero if there is no shell crossing which is reassuring. 

%{\bf For Darsh:}
%
%{\bf I will be happier if this checks out. Please try to go through all the steps and careful about the expansion to 2nd order.}
%
%{\bf The $t_0$ dependent terms do not have to be exactly zero as $t_0=0$ since we have assumed that $t_0\gg d$, when its value is close to $d$, there are some ambiguities. In principle, this can also cover other terms in this order. But I prefer those to be zero on their own.}


\acknowledgments

We thank XXX for useful discussions. 
\appendix

%%%%%%%%%%%%%%%%%%%%%%%%%%%%%%%%%%%%%%%%%%%%%%%%%%%%%%%%%

\bibliographystyle{utcaps}
\bibliography{all_active}

\appendix

\end{document}


